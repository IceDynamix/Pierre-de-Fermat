\section{Vorwort}

\begin{quote}
	Ich habe einen wahrlich wunderbaren Beweis für dieses Problem entdeckt, für den
	dieser Buchrand zu eng ist.
\end{quote}

Das ist das bekannteste Zitat von Fermat, über sein bekannteste Problem, dem großen
Satz von Fermat, ein Problem, welches erst nach 400 Jahren mit modernen Hilfsmitteln
gelöst wurde. Pierre de Fermat war ein hoch-angesehener Mathematiker des 17.
Jahrhunderts. Er hat viele komplizierte Probleme seiner Zeit aus verschiedenen
Themengebieten der Mathematik gelöst und viele Grundsteine der modernen Mathematik
gelegt, darunter Analysis, Stochastik, und, Fermats Lieblingsgebiet, Zahlentheorie.
Und all das, obwohl er Mathematik nur als Hobby gemacht hat und von Beruf her Jurist
war.

In Kapitel \ref{sec:bio} wird sein Leben als Biografie umfasst, sein Studium und sein
Juristenleben werden hier den Mittelpunkt bilden. Dabei wird noch etwas mehr auf
seinen sonderbaren Charakter eingegangen, welchen man als leicht arrogant beschreiben
könnte. Danach wird sich in Kapitel \ref{sec:mathematik} alles rund um seine
Mathematik handeln. Es werden die zuvor erwähnten Themengebiete der Mathematik
angesprochen und ein paar seiner Werke und Lösungen präsentiert. Kapitel
\ref{sec:kurvenanalyse} beschreibt seine Entdeckungen mit Descartes über die
analytische Geometrie, Kapitel \ref{sec:optik} nimmt sich seine Version des
Brechungsgesetz des Lichtes vor. In Kapitel \ref{sec:stochastik} werden zwei Probleme,
welche er mit Pascal gelöst hatte, angesprochen und erklärt. Abschließend werden
vier Probleme der Zahlentheorie beleuchtet, darunter auch seine bekanntesten Probleme,
der kleine und der große Fermatsche Satz.

Wer sich noch nicht viel mit Mathematik beschäftigt hat wird hier auf unbekannte
Begriffe und Symbole treffen, daher werden im Glossar auf Seite \pageref{sec:glossary}
und im Symbolverzeichnis auf Seite \pageref{sec:symb} viele dieser aufzufinden sein.
